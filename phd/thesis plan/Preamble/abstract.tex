\unnumberedchapter{Abstract} 
\chapter*{Abstract} 
\subsection*{\thesistitle}

People with temporal or permanent disabilities who aim to interact with computer based system faces a series of difficulties finding technology that fit their needs. Deafness is known to be the most prevalent disability in the Peruvian population who represents a minority with no equal access to technology and services. 

Recent advances in user experience, usability standards and regulations supported the emergence of inclusive computer systems. However very little has been done to enable user experiences specialized for deaf. This doctoral thesis proposal describes a future effort that combines state of art self supervised learning techniques for unlabeled corpus annotation,  graphical gestural user interfaces to implement functional prototype user interface that enables a fluent communication between deaf and a computing system.

Self supervised learning techniques will be applied to the Peruvian signs language corpus developed by the grammar and signs research group of the Pontifical Catholic University of Peru contributing with future researches on the Signs Language recognition field. Human modeling animation will be applied to the self supervised annotation output for building the graphical gestural user interface.

We believe this work will tremendously contribute with the scientific community specially with the Peruvian community who is actively working on improving and enabling technology access to minorities.