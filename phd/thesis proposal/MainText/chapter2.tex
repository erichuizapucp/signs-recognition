\chapter{Methodology} \label{chap-2}

\section{Available Datasets}\label{available-datasets}
The PSL dataset was developed by the PUCP Grammar and Signs research group in 2014 and consists in a set of videos recorded during the interviews of 24 individuals, 12 male and 12 female informants, all of them are Lima Peru residents and reported to be born with a permanent deafness condition or acquired the condition before the acquisition of Spanish. 

The dataset consists in 718 video clips recorded with a ADR-CX220 SONY HD camera which included an embedded microphone. The camera focused only the informant but also recorded questions, instructions and translations.

The video clips were recorded in three sessions with the following participants: A coordinator, a PSL \cite{lsp_2015} translator and a informant.\\

\textbf{Recording Session 1}: A 45-60 minutes semi structured interview that included: Biographic information as well as habits, anecdotes, opinion about cultural subjects and elicitation of names, states and actions. 

\textbf{Recording Session 2}: The informant was presented with a set of 55 cards describing actions and were asked to choose a set of them in order to build a coherent story that was subsequently told by the informant.

\textbf{Recording Session 3}: A PSL \cite{lsp_2015} conversation facilitated by the coordinator happening between the informant and the translator.

During all the sessions a PSL \cite{lsp_2015} translator performs a translation after a word or phrase is completed.

Other existing Sing Language Recognition are publicly available. These datasets can be characterized as isolated or continuous, taking into account whether annotation are provided at the gloss (fine-grained) or the sentence (coarse-grained) levels. Additionally, they can be divided into Signer Dependent (SD) and Signer Independent (SI) ones, based on the defined evaluation scheme. In particular, in the SI datasets a signer cannot be present in both the training and the test set. In Table \ref{tab:public-large-datasets}, the following most widely known public SLR datasets, along with their main characteristics.
